\documentclass[12pt]{article}
\usepackage{amsmath, amsthm,graphics,setspace,graphicx,color}

\begin{document}

\begin{large}
\noindent Math 1023, Polar Area Lab
\end{large} \newline
\noindent In this assignment we will learn to use the Desmos Graphing Calculator to graph regions using polar coordinates and investigate properties of the graphs. \newline
 
\noindent For each exercise, take a screen shot of your work (the equations and the graphs).  Only do one exercise at a time. Don’t try to put all the exercises in one screen shot. \newline

\noindent Submit a single PDF with your collection of screenshots. \newline

\noindent Do not forget to put your name(s) in the filename as well as in your document. \newline

\noindent (Ex: Smith\_Jones\_lab3.pdf)

\begin{enumerate}
	\item A circular pizza is $10$ inches in diameter with a $2$ inch hole in the middle.  Someone then removes one slice that that represents $\frac{1}{4}$ of the pizza.
	\begin{enumerate}
		\item Determine a range in polar coordinates that describes the remaining pizza.
		\item Using the Desmos Graphing Calculautor, depict a plot of the region.
	\end{enumerate}
	
	\item Consider the polar graph of $\displaystyle r = 5 \cos(3 \theta)$.
	\begin{enumerate}
		\item Using the Desmos Graphing Calculator, estimate the range of polar coordinates that describe the region inside of the upper left petal of $\displaystyle r = 5 \cos(3 \theta)$ and outside of the circle $r = \displaystyle \frac{5}{2}$.
		\item Using your work above and some trigonometry, can you determine analytically the range for $\theta$ for the above region in multiples of $\pi$?

	\end{enumerate}		
	
	
	
\end{enumerate}




\end{document}