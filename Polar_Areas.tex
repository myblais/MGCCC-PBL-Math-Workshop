\documentclass{beamer}
% Make sure to run this file with PDFLatex.
%%%%%%%%%%%%%%%%%%%%%%%%%%%%%
% Do not change the next few lines.
\usepackage{graphicx, amsfonts, amssymb, amsthm}
 \setbeamercolor{alerted text}{fg=magenta}
% \setbeamercolor{alerted text}{fg=blue}
% \setbeamercolor{alerted text}{fg=green}
% \setbeamercolor{alerted text}{fg=yellow}
% \setbeamercolor{alerted text}{fg=cyan}
% \setbeamercolor{alerted text}{fg=gray}
% \setbeamercolor{alerted text}{fg=orange}
% \setbeamercolor{alerted text}{fg=violet}
% \setbeamercolor{alerted text}{fg=purple}
%\setbeamercolor{alerted text}{fg=brown}
% \xdefinecolor{lavendar}{rgb}{0.8,0.6,1}
% \setbeamercolor{alerted text}{fg=lavendar}
\newcommand{\singleslidebig}[1]{\begin{frame}
\begin{center}
\textcolor{blue}{\Huge{#1}}
\end{center}
\end{frame}}
%%%%%%%%%%%%%%%%%%%%%%%%%%%%%
% The beamerthemesplit package is a slide style. You can use it if you like.
% The same with the different themes. You can experiment to see which you like best.
\usepackage{beamerthemesplit}
% \usetheme{default}
% \usetheme{Boadilla}
% \usetheme{Madrid}
% \usetheme{Montpellier}
%\usetheme{Warsaw}

% \usetheme{Copenhagen}
\usetheme{CambridgeUS}

% \usetheme{Goettingen}
% \usetheme{Hannover}
\usetheme{Berkeley}
%%%%%%%%%%%%%%%%%%%%%%%%%%%%%
% Now Start entering your information.
\title{Plane Regions in Polar Coordinates}


\author[Calculus Labs]{{\textbf{Worcester Polytechnic Institute}\\ Department of Mathematical Sciences}}
\institute[WPI]
{Calculus Labs}
\date{}

\begin{document}

\frame{\titlepage}

%%%%%%
%\begin{frame}
%\frametitle{Outline}
%\tableofcontents% Remove the [pausesections] if you want the whole list to show up at once
%\end{frame}
%%%%%%

%%


\begin{frame}
\frametitle{Area Example 1} 

Using polar coordinate inequalities describe the region in the the first quadrant of the $xy$-plane that lies inside of the graph of 

\begin{equation}
	r=\sqrt{\cos \theta}.
\end{equation}

\vspace{0.5 cm}
We plot this region using www.desmos.com/calculator



\vspace{6.5 cm}
\end{frame}





\begin{frame}
\frametitle{Area Depiction}

\begin{figure}[r]
 \raggedleft
%	\begin{center}
	\includegraphics[scale = 0.215]{Plot2.png}
%	\caption{$\int_a^b f(x) dx$} 
%	\end{center}
\end{figure}


\vspace{6.5 cm}
\end{frame}





\begin{frame}
\frametitle{Area Example 2} 

Using polar coordinate inequalities describe the region in the the $xy$-plane that lies inside of the graph of 

\begin{equation}
	r=3\sin \theta
\end{equation}

  and outside of the graph of 

 \begin{equation} 
  	r=1+ \sin \theta.
\end{equation}


\vspace{0.5 cm}
We plot this region using www.desmos.com/calculator



\vspace{6.5 cm}
\end{frame}







\begin{frame}
\frametitle{Area Depiction}


\begin{figure}[r]
 \raggedleft
%	\begin{center}
	\includegraphics[scale = 0.215]{Plot1.png}
%	\caption{$\int_a^b f(x) dx$} 
%	\end{center}
\end{figure}



\vspace{6.5 cm}
\end{frame}






\begin{frame}
\frametitle{Analytic Solution}
We find the intersection of 
\begin{equation}
	r=3\sin \theta
\end{equation}
and
 \begin{equation} 
  	r=1+ \sin \theta.
\end{equation}

$\displaystyle 3 \sin \theta = 1 + \sin \theta$

\vspace{6.5 cm}
\end{frame}



\begin{frame}
\frametitle{Polar Region}


The region in the the $xy$-plane that lies inside of the graph of 

\begin{equation}
	r=3\sin \theta
\end{equation}

  and outside of the graph of 

 \begin{equation} 
  	r=1+ \sin \theta.
\end{equation}
can be described by

\begin{equation} 
   1 + \sin \theta \leq r \leq 	3 \sin \theta  \textrm{ and } \frac{\pi}{6} \leq \theta \leq \frac{5\pi}{6}
\end{equation}

\vspace{6.5 cm}
\end{frame}




\end{document}
