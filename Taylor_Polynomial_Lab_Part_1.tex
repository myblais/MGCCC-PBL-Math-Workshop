\documentclass{beamer}
% Make sure to run this file with PDFLatex.
%%%%%%%%%%%%%%%%%%%%%%%%%%%%%
% Do not change the next few lines.
\usepackage{graphicx, amsfonts, amssymb, amsthm}
 \setbeamercolor{alerted text}{fg=magenta}
% \setbeamercolor{alerted text}{fg=blue}
% \setbeamercolor{alerted text}{fg=green}
% \setbeamercolor{alerted text}{fg=yellow}
% \setbeamercolor{alerted text}{fg=cyan}
% \setbeamercolor{alerted text}{fg=gray}
% \setbeamercolor{alerted text}{fg=orange}
% \setbeamercolor{alerted text}{fg=violet}
% \setbeamercolor{alerted text}{fg=purple}
%\setbeamercolor{alerted text}{fg=brown}
% \xdefinecolor{lavendar}{rgb}{0.8,0.6,1}
% \setbeamercolor{alerted text}{fg=lavendar}
\newcommand{\singleslidebig}[1]{\begin{frame}
\begin{center}
\textcolor{blue}{\Huge{#1}}
\end{center}
\end{frame}}
%%%%%%%%%%%%%%%%%%%%%%%%%%%%%
% The beamerthemesplit package is a slide style. You can use it if you like.
% The same with the different themes. You can experiment to see which you like best.
\usepackage{beamerthemesplit}
% \usetheme{default}
% \usetheme{Boadilla}
% \usetheme{Madrid}
% \usetheme{Montpellier}
%\usetheme{Warsaw}

% \usetheme{Copenhagen}
\usetheme{CambridgeUS}

% \usetheme{Goettingen}
% \usetheme{Hannover}
\usetheme{Berkeley}
%%%%%%%%%%%%%%%%%%%%%%%%%%%%%
% Now Start entering your information.
\title{Taylor Polynomials}


\author[Calculus Labs]{{\textbf{Worcester Polytechnic Institute}\\ Department of Mathematical Sciences}}
\institute[WPI]
{Calculus Labs}
\date{}

\begin{document}

\frame{\titlepage}

%%%%%%
%\begin{frame}
%\frametitle{Outline}
%\tableofcontents% Remove the [pausesections] if you want the whole list to show up at once
%\end{frame}
%%%%%%

%%


\begin{frame}
\section{Taylor Series}
\frametitle{Taylor Series}
\begin{definition} Let $f$ be a function with derivatives of all orders throughout some interval containing $a$.  \newline The \begin{bf}Taylor series\end{bf} generated by $f$ at $x=a$ is
$\begin{array}{l}
		\\
	\displaystyle \sum_{n=0}^\infty \frac{f^{(n)}(a)}{n!}(x-a)^n \\
		\\
	\displaystyle  = f(a) + f'(a)(x-a) + \frac{f''(a)}{2}(x-a)^2 + \frac{f'''(a)}{3!}(x-a)^3 + \dots. \\
\end{array}$
\end{definition}

\vspace{5.5 cm}
\end{frame}




\begin{frame}
\section{Taylor Polynomials}
\frametitle{Taylor Polynomial}
\begin{definition}
Let $f$ be a function with derivatives of order $n$ for \newline $n=1,2,\dots,N$ throughout some interval containing $a$.  Then for any integer $K \in \{0,1,2,\dots,N\}$ the \begin{bf}Taylor polynomial of order $K$\end{bf} generated by $f$ at $x=a$ is the polynomial
\begin{equation}
	P_K(x) = \sum_{n=0}^K \frac{f^{(n)}(a)}{n!}(x-a)^n ,
\end{equation}
or
\begin{equation}
	P_K(x)  = f(a) + f'(a)(x-a) + \frac{f''(a)}{2}(x-a)^2 + \dots + \frac{f^{(K)}(a)}{K!}(x-a)^K.
\end{equation}
\end{definition}

\vspace{5.5 cm}
\end{frame}











\begin{frame}
\frametitle{Example: Maclaurin Series for $\sin(bx)$} 
$\displaystyle f(x) = \sin(bx)$ has Maclaurin series:\newline

$\displaystyle \sin(u) =  \sum_{n=0}^\infty (-1)^n \frac{u^{2n+1}}{(2n+1)!}$\newline

$\displaystyle = u-\frac{u^3}{3!} + \frac{u^5}{5!} - \frac{u^7}{7!}+\frac{u^9}{9!}- \dots$  for all $u$ in $(-\infty, \infty)$ \newline

Substitute $u=bx$

$\displaystyle \sin(bx) =  \sum_{n=0}^\infty (-1)^n \frac{(bx)^{2n+1}}{(2n+1)!}$ \newline

\vspace{0.5 cm}
$\displaystyle = bx-\frac{b^3x^3}{3!} + \frac{b^5x^5}{5!} - \frac{b^7x^7}{7!}+\frac{b^9x^9}{9!}- \dots$ \newline



\vspace{6.5 cm}
\end{frame}







\begin{frame}
\frametitle{Maclaurin Polynomials for $\sin(bx)$}
\section{Example}

$\displaystyle \sin(bx) =  \sum_{n=0}^\infty (-1)^n \frac{b^{2n+1}}{(2n+1)!} \cdot x^{2n+1} $ \newline

\vspace{0.5 cm}
$\displaystyle = bx-\frac{b^3}{3!}x^3 + \frac{b^5}{5!}x^5 - \frac{b^7}{7!}x^7+\frac{b^9}{9!}x^9- \dots$ \newline
\vspace{0.3 cm}

$\displaystyle T_1(x) = $ \newline
\vspace{0.1 cm}

$\displaystyle T_3(x) = $ \newline
\vspace{0.1 cm}

$\displaystyle T_5(x) = $ \newline
\vspace{0.1 cm}

$\displaystyle T_7(x) = $ \newline
\vspace{6.5 cm}
\end{frame}





\begin{frame}
\frametitle{Maclaurin Polynomials for $\sin(bx)$}


$\displaystyle T_1(x) =bx$ \newline
\vspace{0.2 cm}

$\displaystyle T_3(x) = bx-\frac{b^3}{3!}x^3 $ \newline
\vspace{0.2 cm}

$\displaystyle T_5(x) =bx-\frac{b^3}{3!}x^3 + \frac{b^5}{5!}x^5 $ \newline

\vspace{0.2 cm}

$\displaystyle T_7(x) =bx-\frac{b^3}{3!}x^3 + \frac{b^5}{5!}x^5  - \frac{b^7}{7!}x^7$ \newline

\vspace{0.6cm}

Graph in Desmos: https://www.desmos.coms

\vspace{6.5 cm}
\end{frame}



\begin{frame}
\frametitle{Approximations to $\sin(2x)$ at $x = 1.5$}

$\displaystyle T_1(1.5) = 3$ \newline
\vspace{0.2 cm}

$\displaystyle T_3(1.5) = -1.5 $ \newline
\vspace{0.2 cm}

$\displaystyle T_5(x) =0.525 $ \newline

\vspace{0.2 cm}

$\displaystyle T_7(x) =0.0911$ \newline

\vspace{0.6cm}

\vspace{1 cm}

$\sin(2 \cdot 1.5) \approx 0.14112001$

\vspace{6.5 cm}
\end{frame}



\begin{frame}
\frametitle{Errors of Approximations to $\sin(2x)$ at $x = 1.5$}

\begin{equation}
	|\textrm{Error}| = |\textrm{True Value} - \textrm{Approximation}|
\end{equation}
\vspace{0.7 cm}

\begin{tabular}{ll}
	\vspace{0.2 cm}
	\hspace{0.01 cm}Polynomial Approximation & \hspace{0.3 cm} Error \\
	\vspace{0.2 cm}
	$T_1(1.5)$ 	&	$|\sin(2 \cdot 1.5) - T_1(1.5) |= 2.8589$ 	\\
	\vspace{0.2 cm}
	$T_3(1.5)$ 	&	$|\sin(2 \cdot 1.5) - T_3(1.5) |= 1.6411$ 	\\
	\vspace{0.2 cm}
	$T_5(1.5)$ 	&	$|\sin(2 \cdot 1.5) - T_5(1.5) |= 0.3839$ 	\\
	\vspace{0.2 cm}
	$T_7(1.5)$ 	&	$|\sin(2 \cdot 1.5) - T_7(1.5) |= 0.0500$ 	\\
\end{tabular}.




\vspace{6.5 cm}
\end{frame}




\end{document}
